%%%%%%%%%%%%%%%%%%%%%%%%%%%%%%%%%%%%%%%%%
% baposter Portrait Poster
% LaTeX Template
% Version 1.0 (15/5/13)
%
% Created by:
% Brian Amberg (baposter@brian-amberg.de)
%
% This template has been downloaded from:
% http://www.LaTeXTemplates.com
%
% License:
% CC BY-NC-SA 3.0 (http://creativecommons.org/licenses/by-nc-sa/3.0/)
%
%%%%%%%%%%%%%%%%%%%%%%%%%%%%%%%%%%%%%%%%%

\documentclass[a0paper,portrait]{baposter}
\usepackage[utf8]{inputenc} % input encoding : utf8, latin9, latin1
\usepackage[french]{babel} % document language : typography + titles...
\usepackage[T1]{fontenc} % font encoding : ö = single letter and not o + accent

\usepackage{tikz}
\usetikzlibrary{shapes,arrows}
\fboxsep=0mm%padding thickness
\fboxrule=4pt%border thickness

\usepackage{graphicx}
%\usepackage{caption}


\usepackage[font=small,labelfont=bf]{caption} % Required for specifying captions to tables and figures
\usepackage{subcaption}
\usepackage{booktabs} % Horizontal rules in tables
\usepackage{relsize} % Used for making text smaller in some places

\graphicspath{{figures/}} % Directory in which figures are stored

\definecolor{bordercol}{RGB}{40,40,40} % Border color of content boxes
\definecolor{headercol1}{RGB}{186,215,230} % Background color for the header in the content boxes (left side)
\definecolor{headercol2}{RGB}{80,80,80} % Background color for the header in the content boxes (right side)
\definecolor{headerfontcol}{RGB}{0,0,0} % Text color for the header text in the content boxes
\definecolor{boxcolor}{RGB}{186,215,230} % Background color for the content in the content boxes

\begin{document}

\background{ % Set the background to an image (background.pdf)
    \begin{tikzpicture}[remember picture,overlay]
        \draw (current page.north west)+(-2em,2em) node[anchor=north west]
        {\includegraphics[height=1.1\textheight]{background}};
    \end{tikzpicture}
}

\begin{poster}{
        grid=false,
        borderColor=bordercol, % Border color of content boxes
        headerColorOne=headercol1, % Background color for the header in the content boxes (left side)
        headerColorTwo=headercol2, % Background color for the header in the content boxes (right side)
        headerFontColor=headerfontcol, % Text color for the header text in the content boxes
        boxColorOne=boxcolor, % Background color for the content in the content boxes
        headershape=roundedright, % Specify the rounded corner in the content box headers
        headerfont=\Large\sf\bf, % Font modifiers for the text in the content box headers
        textborder=rectangle,
        background=user,
        headerborder=open, % Change to closed for a line under the content box headers
        boxshade=plain
    }
    {}
    %
    %
    {\sf\bf Investigation of LiDAR-based place recognition in forest}
    %\\ \large Preliminary work on navigation in deformable environments} % Poster title
    {\vspace{0.2em} Sébastien Michaud$^1$, Jean-François Lalonde$^2$, Philippe Giguère$^1$\\* % Author names
        {\vspace{-0.4em}\small $^1$Department of Computer Science and Software Engineering, Laval University}\\*
        {\vspace{-0.2em}\small $^2$Department of Electrical Engineering and Computer Engineering, Laval University}}
    {\includegraphics[scale=0.12]{./figures/logo.png}} % University/lab logo

    \headerbox{Problematic}{name=introduction,column=0,row=0}{
        \begin{itemize}
            \item[•] Localization and mapping algorithms require the identification of previously seen places.
            \item[•] Currently available 3D data features might not be reliable for unstructured outdoor environments.
            \item[•] Many objects in forests deform or change over time and can therefore make place recognition challenging.
        \end{itemize}

    }

    \headerbox{Research Question}{name=question,column=0,below=introduction}{
        How to use 3D range data to recognize previously visited places in forested environments?
    }

    \headerbox{Hypothesis}{name=hypothesis,column=0,below=question}{
        Using forest specific 3D features and restricting the search space to regions that are stable over time could improve place recognition results.
    }

    \headerbox{Nomenclature}{name=nomenclature,column=0,below=hypothesis}{
        \textbf{Place recognition:} Association of a new 3D acquisition to a previous scan.\\
        \textbf{[3D] Feature:} Representation of a subset of the 3D data that is relevant to solve the place recognition problem.\\
        \textbf{Bag of words (BoW):} Histogram representation of the 3D features count for a given scan.
        Representation of a scan by the histogram of the 3D features counts, disregarding spatial configuration.\\
        \textbf{Unstructured environments:} Environments in which no predefined structure is available to guide the robot task.
    }

    \headerbox{Materials}{name=material,column=0,below=nomenclature}{
        \textbf{Platform:} Husky A200 \\
        \textbf{LiDAR:} SICK LMS151 or Velodyne HDL-32E\\
        \textbf{Others:} FLIR Motion Control Systems Pan-Tilt Unit-D46-17, CH Robotics UM6 inertial measurement unit
    }

    \headerbox{References}{name=references,column=0,below=material}{
        \smaller % Reduce the font size in this block
        \renewcommand{\section}[2]{\vskip 0.05em} % Get rid of the default "References" section title
        \nocite{*} % Insert publications even if they are not cited in the poster

        \bibliographystyle{abbrv}
        \bibliography{bibliography}
    }

    \headerbox{Acknowledgements}{name=acknowledgements,column=0,below=references, above=bottom}{
        \smaller
        We would like to thank \textit{Fonds de recherche - Nature et technologie} (FRQNT) and \textit{Natural Sciences and Engineering Research Council of Canada} (NSERC).
    }

    \headerbox{Framework base algorithm \cite{Steder2002}}{name=mainSolution,span=2,column=1,row=0}{
        \begin{enumerate}
            \item Build a dictionary (i.e. set of features) from all 3D scans.
            \item Use BoW approach to get initial similarity measure for all scans in the database.
            \item For each ordered correspondence of the scan, calculate a set of possibles transformations between the scan and all previous acquisition.
            \item Score each of the possible transformation and keep the transformation with the highest score. If score > treshold : candidate for recognized place
            \item *. 3-4 until timeout or iteration maxed
        \end{enumerate}
    }

    \headerbox{Find a section title}{name=conceptualSolution,span=2,column=1,below=mainSolution,above=bottom}{ % To reduce this block to 1 column width, remove 'span=2'
        \textbf{3D features : ISS, NARF, Spin Images, 3D SIFT, SPOT, FPFH, VFH, RIFT}
        \begin{itemize}
            \item[•] Write what I would like to investigate (features reliability, point cloud density, region of interest aka tree trunks...)
            \item[•] Velodyne vs SICK
            \item[•] Available Point cloud features
        \end{itemize}

        \textbf{Dataset acquisition path (left) and sample point cloud (right)}
        \begin{center}
            \includegraphics[width=0.455\linewidth]{./figures/path2.png}
            \includegraphics[width=0.48\linewidth]{./figures/sick.png}\\*
            Source: Google Maps (2015)
        \end{center}
    }

    %----------------------------------------------------------------------------------------

\end{poster}

\end{document}
